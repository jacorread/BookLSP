\chapter{La reducción fonética de las vocales}
Add content in chapters/02.tex

nucleares o las sílabas focalizadas. En griego, una lengua que tiene un sistema vocálico muy parecido al español, la duración alcanza sus valores más altos cuando la vocal es tónica, focalizada y se encuentra en posición final de frase, y alcanza sus valores mínimos cuando la vocal es átona y no está asociada con una sílaba entonativamente prominente (Nicoladis, 2003). En español se han observado tendencias semejantes, pero, además, se ha encontrado que la posición de la vocal dentro de la palabra y la presencia de una pausa también modifican la duración y los formantes vocálicos. Gendrot, Adda-Decker y Santiago (2019) encontraron que en español y francés las sílabas iniciales de palabra presentan una duración menor y una f0 más baja que las tónicas finales de palabra. En ambas lenguas, la presencia de una pausa después de la vocal supone valores de frecuencia más altos de los formantes y un espacio acústico periférico. Estos estudios muestran claramente que la articulación de vocales y consonantes se ven reforzados e interactúan con factores prosódicos de mayor nivel. Como sugieren Keating et al. (2004, p. 16), la estructura lingüística se proyecta en los detalles fonéticos y los constituyentes prosódicos pueden manifestarse en detalles articulatorios como la articulación de vocales y consonantes.

\section{Propiedades léxicas}

Las teorías fonológicas actuales han demostrado que hay una relación estrecha entre el contenido semántico del léxico, su predictibilidad en la lengua y los fenómenos de variación y cambio fonético. Un fenómeno bien documentado es la tendencia de las palabras funcionales al debilitamiento de sus fonemas y sílabas. La variabilidad de estas palabras se explica porque no están asociadas al contenido léxico, tienen funciones sintácticas, alta frecuencia de aparición y, por tanto, son unidades predecibles dentro de la cadena de habla. En cambio, las palabras de contenido léxico, «asociadas a conceptos concretos e ideas que pueden ser evocadas» (Bosque, 1990, p.30), se pronuncian con mayor precisión, tienen baja frecuencia de aparición y son menos predecibles (Zipf, 1929; Koopmans-van Beinum y Harder, 1982; Rehor y Pätzold, 1996; Meunier y Espesser, 2011). De acuerdo con Lieberman (1963), los correlatos acústicos de las palabras funcionales son secundarios porque, aun cuando la reducción fonética sea extrema, el oyente puede identificar la información con ayuda del contexto (p.184). En cambio, las palabras con contenido léxico, y su sílaba tónica en particular, deben articularse claramente para que el oyente identifique fácilmente la información léxica (van Bergen, 1993; de Jong, 1995).

Los estudios fonéticos que han comparado palabras funcionales y palabras de contenido léxico concuerdan en que la duración y los formantes vocálicos están determinados por el tipo de palabra o la categoría morfológica a la que pertenece el sonido. Por ejemplo, Engstrand y Krull (1988, pp. 40-42) compararon la secuencia /før/ del sueco cuando hace parte de una preposición y cuando hace parte de un prefijo, y observaron que, en el prefijo, la vocal /ø/ tiene una duración más breve y el segundo formante es más centralizado que cuando la vocal aparece en la preposición. En la misma línea, pero con un resultado contrario, Aguilar, Machuca y Martínez (1991) estudiaron la secuencia /de/ del español como preposición y como sílaba átona de una palabra de contenido. Las autoras encontraron diferencias significativas en la duración de los sonidos /d/ y /e/ asociadas al tipo de palabra, y un F2 más bajo (más centralización) cuando /e/ pertenecía a la preposición. En algunas lenguas, el contenido semántico de la palabra tiene mayor influencia en el timbre vocálico que el acento léxico: en neerlandés, por ejemplo, las palabras funcionales producidas con acento primario 